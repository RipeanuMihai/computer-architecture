
\begin{frame}
\frametitle{Notare}

\begin{itemize}
    \item 40p examen final
    \item 60p Parcurs
    \begin{itemize}
        \item 10p activitate laborator
        \begin{itemize}
            \item 5p prezență
            \item 5p activitate
        \end{itemize}
        \item 10p Parțial teoretic
        \item 10p Colocviu
        \item 30p Proiect
        \begin{itemize}
            \item 10p Documentație
            \item 10p Implementare
            \item 10p Evaluare Sinteză
        \end{itemize}
    \end{itemize}
\end{itemize}
\end{frame}

\begin{frame}
\frametitle{Promovare}
    
\begin{itemize}
    \item UNSTPB/NUSTPB
    \begin{itemize}
        \item Minim 50p Total
    \end{itemize}
\end{itemize}
\end{frame}

\begin{frame}
\frametitle{Parțial Teoretic}

\begin{itemize}
    \item Materia necesară: laboratoarele 1-6
    \item Durată: 60 de minute
    \item Când: la începutul laboratorului 7
    \item Structură: 20 întrebări teoretice
    \item Locație: sala de laborator
    \item Platformă: Calculator Laborator - Moodle - Quiz
\end{itemize}
\end{frame}

\begin{frame}
\frametitle{Colocviu}

\begin{itemize}
    \item Materia necesară: toate laboratoarele
    \item Durată: 120 de minute
    \item Când: la ultimul laborator
    \item Structură:
    \begin{itemize}
        \item exerciții practice de implementat în Verilog
    \end{itemize}
    \item Locație: sala de laborator
    \item Platformă: Calculator Laborator - Moodle - VPL/Quiz
\end{itemize}
\end{frame}

\begin{frame}
\frametitle{Proiect}

\begin{itemize}
    \item Se lucrează în echipe de câte 3 studenți
    \item Teme propuse de echipa de asistenți în a doua săptămână de laborator
    \item Deadline documentație: 05.11.2025, 23:59 - Latex/Typst/Markdown - GitHub
    \item Deadline implementare: 17.12.2025, 23:59 - Verilog/RHDL/Chisel/VHDL - GitHub
    \item Deadline evaluare: 14.01.2026, 23:59 - GitHub
\end{itemize}
\end{frame}

\begin{frame}
\frametitle{Examen Final (AB)}

\begin{itemize}
    \item Platformă: Moodle - Quiz
    \item Durată: 120 de minute
    \item Structură:
    \begin{itemize}
        \item 20 de întrebări teoretice
    \end{itemize}
\end{itemize}
\end{frame}

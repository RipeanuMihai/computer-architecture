
\begin{frame}
    \frametitle{Parcurs academic}
    \begin{itemize}
        \item Licență: UPB-ACS/IPP-ENSTA "Interacțiunea om-robot într-un context de asistență socială"
        \item Disertație: UPB-ACS/IPP-ENSTA "Social engineering attack using humanoid robots"
        \item Doctorat: UPB-ACS/NUS-SOC "Number representation systems in computer engineering"
        \item Post-Doctorat: NUS-SOC "Singapore Blockchain Innovation Programme"
        \item Proiecte: CuEVM \footnotemark[1], NRSL/HNRSL, ENRICHME
        \item Interese academice: arhitectura calculatoarelor, matematică, blockchain, criptografie
    \end{itemize}
    \note{
    }
    \footnotetext[1]{https://github.com/sbip-sg/CuEVM}
\end{frame}

\begin{frame}
    \frametitle{Parcurs didactic}
    \begin{itemize}
        \item Instructor Hackademy: CCNA(2014-2016), C++ (2015)
        \item Tutor UPB-ACS (2015-2018): Calculatoare numerice 1 (2015), Calculatoare numerice 2 (2016), Procesarea semnalelor (2017), Rețele locale (2017), Proiectarea rețelelor (2017).
        \item Asistent UPB-ACS (2019-2021): Arhitectura calculatoarelor (2019), Calculatoare numerice 1 (2020), Calculatoare numerice 2 (2020), Arhitectura Sistemelor de calcul (2022),  Rețele locale (2019-2020), Proiectarea rețelelor (2019-2021), Proiectarea rețelelor neurale private(2022), Tehnici de protecție a vieții private (2019-2021).
        \item Asistent NUS-SOC (2022-2023): Cloud Computing (2022)
        \item Guest Lecturer NUS-SOC (2023-2024): Cloud Computing (2023)
        \item Școli de vară : UPB-ACS Arhitectura calculatoarelor (2022), UPB-ACS Data Science (2022), NUS-SOC Cloud Computing (2023)
        \item Interese didactice: open source, corectare automată
    \end{itemize}
    \note{
    }
\end{frame}

\begin{frame}
    \frametitle{Cuprins materie}
    \begin{enumerate}
        \setcounter{enumi}{0}
        \item Introducere
        \item Structură Calculator Numeric
        \item Reprezentarea și prelucrarea informației
        \item Memorii
        \item UAL
        \item Arhitectura Calculatorului Didactic
        \item Limbaje de asamblare
        \item Subsistem intrare/ieșire
        \item Întreruperi
        \item Microprogramare
    \end{enumerate}
\end{frame}
    
\begin{frame}
    \frametitle{Obiectivele materiei}
    
    \begin{itemize}
        \item Cultură generală despre arhitectura calculatoarelor
        \begin{itemize}
            \item Reprezentarea informației
            \item Memorii
            \item Seturi de instrucțiuni
            \item Codificare instrucțiuni
            \item Limbaj de asamblare
            \item Întreruperi de procesor
        \end{itemize}
        \item Descriere hardware a unui procesor prin limbajul Verilog
    \end{itemize}
\end{frame}




\begin{frame}
    \frametitle{Platforma cursului}
    
    \begin{itemize}
        \item \href{https://github.com/cs-pub-ro/computer-architecture}{GitHub: https://github.com/cs-pub-ro/computer-architecture}
        \item \href{https://cs-pub-ro.github.io/computer-architecture/}{Open Education Hub: https://cs-pub-ro.github.io/computer-architecture/}
        \item \href{https://curs.upb.ro/2024/course/view.php?id=1815}{Moodle: https://curs.upb.ro/2024/course/view.php?id=1815}
    \end{itemize}
\end{frame}
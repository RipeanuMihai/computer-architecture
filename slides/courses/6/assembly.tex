\begin{frame}
    \frametitle{Assembly Language}
    \begin{itemize}
        \item Low-level programming language
        \item Directly related to machine code
        \item Most cases: each instruction corresponds to a machine code instruction
    \end{itemize}
\end{frame}

\begin{frame}
    \frametitle{Advantages}
    \begin{itemize}
        \item Direct access to hardware
        \item High performance
        \item Small code size
        \item Full control over the system
    \end{itemize}
\end{frame}

\begin{frame}
    \frametitle{Disadvantages}
    \begin{itemize}
        \item Difficult to write
        \item Difficult to read
        \item Difficult to maintain
        \item Difficult to debug
        \item Difficult to port
        \item Time consuming
    \end{itemize}
\end{frame}


\begin{frame}
    \frametitle{Position of Code}
    \begin{itemize}
        \item Independent (absolute/PIC)
        \item Relative to the current instruction (PC-relative)
        \item External to the code (libaries)
    \end{itemize}
\end{frame}


\begin{frame}
    \frametitle{Mnemonics}
    \begin{itemize}
        \item Instruction:
        
        [Label:] \texttt{Operation} [List of Operands] [; Comment]

        \item Variables:
        
        \texttt{name type size} [value/expresion] [; Comment]

    \end{itemize}
\end{frame}

\begin{frame}
    \frametitle{Data Types}
    \begin{itemize}
        \item DB Byte: 8 bits
        \item DW Word: 16 bits
        \item DD Double Word: 32 bits
        \item DQ Quad Word: 64 bits
        \item DT Ten Bytes: 80 bits
    \end{itemize}
\end{frame}

\begin{frame}
    \frametitle{Constants}
    \begin{itemize}
        \item binary 10101B
        \item octal 1234Q
        \item decimal 1234D
        \item hexadecimal 1234H (first digit must be 0-9)
        \item character 'A'
    \end{itemize}
\end{frame}

\begin{frame}
    \frametitle{Operator DUP}
    \begin{itemize}
        \item \texttt{var DB 10 DUP(?)}
        \item \texttt{var DB 10 DUP(0)}
        \item \texttt{var DB 5 DUP(5 DUP(5 DUP(0)))}
    \end{itemize}
\end{frame}

\begin{frame}
    \frametitle{Sections}
    \begin{itemize}
        \item \texttt{.data}
        \item \texttt{.text}
        \item \texttt{.rodata}
        \item \texttt{.section}
        \item \texttt{.global}
        \item \texttt{.extern}
    \end{itemize}
\end{frame}

\begin{frame}
    \frametitle{Directives}
    \begin{itemize}
        \item \texttt{.equ}
        \item \texttt{=}
        \item \texttt{.asci}
        \item \texttt{.include}
    \end{itemize}
\end{frame}


\begin{frame}[fragile]
    \frametitle{Code Example}
    \begin{minted}{gas}
    VEC DW 10 DUP(0); 10 words initialized with 0
    SUM DW 0; 1 word initialized with 0
    SIZE EQU 10; SIZE = 10

    MAIN:
        MOV BA, VEC; BA = VEC
        XOR XA, XA; XA = 0

        INIT_LOOP:
            MOV [BA + XA+], XA; VEC[XA++] = XA
            CMP XA, 10; Compare XA with 10
            JL INIT_LOOP; If XA < 10, jump to INIT_LOOP
        
    \end{minted}
    
\end{frame}


\begin{frame}[fragile]
    \frametitle{Code Example}
    \begin{minted}{gas}
    
        MOV BA, VEC; BA = VEC
        XOR XA, XA; XA = 0
        MOV RA, XA; RA = XA

        SUM_LOOP:
            ADD RA, [BA + XA+]; RA += VEC[XA++]
            CMP XA, 10; Compare XA with 10
            JL SUM_LOOP; If XA < 10, jump to SUM_LOOP

        MOV SUM, RA; SUM = RA
        HALT:
            HLT; Stop the program
        
    \end{minted}
    
\end{frame}
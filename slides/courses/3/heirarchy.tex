\begin{frame}
    \frametitle{Memory Hierarchy}
    \begin{columns}
        \column{0.5\textwidth}
        \begin{itemize}
            \item The CPU makes requests to memory for specific addresses.
            \begin{itemize}
                \item Hit: Data is found.
                \item Miss: Data is not found.
            \end{itemize}
            \item For cache memory:
            \begin{itemize}
                \item Cache hit, cache miss.
                \item Hardware manages cache levels (L1, L2, L3).
                \item Uses blocks or lines as units of transfer.
            \end{itemize}
            \item For virtual memory:
            \begin{itemize}
                \item Page hit, page fault.
                \item The operating system manages virtual memory.
                \item Uses pages as units of transfer.
            \end{itemize}
        \end{itemize}

        \column{0.5\textwidth}
        \newsavebox{\asciimemheir}
        \begin{lrbox}{\asciimemheir}
            \begin{varwidth}{\maxdimen}
            \VerbatimInput[fontsize=\scriptsize]{media/memheir.ascii}
            \end{varwidth}
        \end{lrbox}%

        \begin{figure}[h]
            \centering
            \scalebox{0.7}{\usebox{\asciimemheir}}
        \end{figure}

    \end{columns}
    \note{
        The memory hierarchy creates the illusion of a large, fast memory system.
    }
\end{frame}

\begin{frame}
    \frametitle{Memory Locality}
    The memory hierarchy creates the illusion of a large, fast memory system.
    \begin{itemize}
        \item \textbf{Temporal locality:} If a memory location is accessed, it is likely to be accessed again soon.
        \item \textbf{Spatial locality:} If a memory location is accessed, nearby memory locations will likely be accessed soon after.
    \end{itemize}
    A full cache memory miss depends on two factors: the latency of the main memory (how long it takes to retrieve the first byte) and the bandwidth (how long it takes to retrieve the entire line or block).
\end{frame}

\begin{frame}
    \frametitle{CPU Execution Time}
    \begin{itemize}
        \item \textbf{CPU execution time} = CPU clock cycles $\times$ Clock cycle time (CLKT)
        \item \textbf{CPU clock cycles} = Instruction count (IC) $\times$ Cycles per instruction (CPI)
        \item \textbf{With memory hierarchy:} CPU clock cycles = (CPU clock cycles $+$ Memory stall cycles) $\times$ Clock cycle time
        \item \textbf{Memory stall cycles} = Number of misses $\times$ Miss penalty
        \item \textbf{Memory stall cycles} = IC $\times$ Misses per instruction $\times$ Miss penalty
        \item \textbf{Misses per instruction} = Miss rate $\times$ Memory accesses per instruction (this can be separated for read and write operations)
    \end{itemize}
\end{frame}
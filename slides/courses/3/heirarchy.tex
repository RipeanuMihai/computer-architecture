\begin{frame}
    \frametitle{Memory Hierarchy}
    \begin{columns}
        \column{0.5\textwidth}
        \begin{itemize}
            \item CPU makes requests to memory for specific addresses.
            \begin{itemize}
                \item Hit: Data found.
                \item Miss: Data not found.
            \end{itemize}
            \item For cache memory:
            \begin{itemize}
                \item cache hit, cache miss.
                \item Hardware manages cache. (L1, L2, L3)
                \item Use block/line as a unit of transfer.
            \end{itemize}
            \item For virtual memory:
            \begin{itemize}
                \item page hit, page fault (pault)
                \item OS manages virtual memory.
                \item Use page as a unit of transfer.
            \end{itemize}
            \item Memory hierarchy creates the illusion of a large, fast memory.
        \end{itemize}

        \column{0.5\textwidth}
        \newsavebox{\asciimemheir}
        \begin{lrbox}{\asciimemheir}
            \begin{varwidth}{\maxdimen}
            \VerbatimInput[fontsize=\scriptsize]{media/memheir.ascii}
            \end{varwidth}
        \end{lrbox}%

        \begin{figure}[h]
            \centering
            \scalebox{0.7}{\usebox{\asciimemheir}}
        \end{figure}

    \end{columns}
    \note{
    Memory hierarchy creates the illusion of a large, fast memory.
    }
\end{frame}

\begin{frame}
    \frametitle{Memory Locality}
    \begin{itemize}
        \item Temporal locality: If a memory location is referenced, it will tend to be referenced again soon.
        \item Spatial locality: If a memory location is referenced, nearby memory locations will tend to be referenced soon.
    \end{itemize}
    A full cache memory miss depends on the main memory's latency (how long it takes to get the first byte) and bandwidth (how long it takes to get the entire line/block).
\end{frame}

\begin{frame}
    \frametitle{CPU execution time}
    \begin{itemize}
        \item CPU execution time = CPU clock cycles * Clock cycle time (CLKT)
        \item CPU clock cycles = Instruction count (IC) * CPI
        \item CPI = Cycles per instruction
        \item With memory hierarchy: CPU clock cycles = (CPU clock cycles + Memory stall cycles) * Clock cycle time
        \item Memory stall cycles = Number of misses * Miss penalty
        \item Memory stall cycles = IC * Misses per instruction * Miss penalty
        \item Misses per instruction = Miss rate * Memory access per instruction (can be separated for read and write)
    \end{itemize}
\end{frame}
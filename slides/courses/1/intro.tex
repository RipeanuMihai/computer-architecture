
\begin{frame}
    \frametitle{Informația}
    \begin{itemize}
        \item Structură și reprezentare
        \item Stocare
        \item Prelucrare
        \item Transmitere
    \end{itemize}
    \note{
    }
\end{frame}

\begin{frame}
    \frametitle{Reprezentare}
    BCD - Binary Coded Decimal
    \begin{itemize}
        \item 4 biti pentru fiecare cifra (10 valori)
        \item 0-9
    \end{itemize}
    Integer - 2's complement
    \begin{itemize}
        \item 4 biți reprezintă 16 valori
        \item Mai eficient decât BCD conform Shannon
    \end{itemize}
    \note{
    }
\end{frame}

\begin{frame}
    \frametitle{Calculator Analogic}
    \begin{itemize}
        \item BCD pe 10 intervale de tensiune
        \item Operați aritmetice pe tensiune
    \end{itemize}
\end{frame}
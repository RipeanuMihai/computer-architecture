\begin{frame}
    \frametitle{Niveluri funcționale ale unui CN}
\begin{enumerate}
    \item Dispozitve și circuite electronice (Transiztoare, portți logice) - hardware
    \item Unități funcționale (UAL, Memorie, Interfețe) - firmware (Micropgramare) + hardware
    \item Mașină fizică Strcuturi de interconectare și echipamente periferice - hardware
    \item Nucleul sistem de operare - BIOS - firmware
    \item Sistem de Operare și API-ul său - software
    \item Limbaj de programare și compilator - software
    \item Aplicații și biblioteci de programare - software
    \item Interfața cu utilizatorul - software
\end{enumerate}

\end{frame}

\begin{frame}
    \frametitle{Hardware vs Software}
\begin{itemize}
    \item Anumite funcții pot fi implementate atât în hardware cât și în software
    \item Decizia de a implementa o funcție în hardware sau software se bazează pe:
    \begin{itemize}
        \item Costul implementării
        \item Viteza de execuție
        \item Raport cost/performanță
        \item Siguranță în funcționare
        \item Frecvența unor modificări
    \end{itemize}
    \item Tendința actuală este de a trece cât mai multe funcții în hardware
\end{itemize}

\end{frame}
\subsection{ADD/SUB}

\begin{frame}
    \frametitle{ADD/SUB}
    A selector cand be done such that any of $ADD$ or $SUB$ can be performend with the same hardware in the case of two's complement.
    Let's consider the following notations for the operands $x$ and $y$.
    \begin{equation}
        \begin{aligned}
            &result=
                \begin{cases}
                    x+y,& \text{if ADD}\\
                    x+(-y), & \text{if SUB}
                \end{cases}
        \end{aligned}
    \end{equation}
\end{frame}

\begin{frame}
    \frametitle{ADD/SUB}
\end{frame}

\begin{frame}
    \frametitle{Adder}
    Type of adder implementations in hardware are:
    \begin{itemize}
        \item Ripple Carry Adder
        \item Carry Lookahead Adder
            \begin{itemize}
                \item Carry Skip Adder (bypass)
                \item Carry Select Adder (multiplexer)
            \end{itemize}
    \end{itemize}
\end{frame}

\subsection{MUL}

\subsection{DIV/MOD}

\subsection{ROOT}

\subsection{POWER}

\section*{Problem}
Having the following UART configuration, what is the maximum data rate that can be achieved?
\begin{itemize}
    \item Baud rate: 115200
    \item Data bits: 8
    \item Stop bits: 1
    \item Parity: None
    \item Flow control: None
\end{itemize}

\section*{Solution}
To determine the maximum data rate that can be achieved with the given UART configuration, we need to consider the baud rate and the number of bits transmitted per frame.

1. Baud Rate:
\[
\text{Baud rate} = 115200 \text{ bits per second (bps)}
\]

2. Frame Structure:
\begin{itemize}
    \item Data bits: 8
    \item Stop bits: 1
    \item Parity bit: None
\end{itemize}

Each frame consists of:
\[
\text{Total bits per frame} = \text{Start bit} + \text{Data bits} + \text{Parity bit} + \text{Stop bits}
\]
\[
\text{Total bits per frame} = 1 + 8 + 0 + 1 = 10 \text{ bits}
\]

3. Maximum Data Rate:
The maximum data rate is determined by the number of data bits transmitted per second. Given the baud rate and the frame structure:
\[
\text{Maximum data rate} = \frac{\text{Baud rate} \times \text{Data bits}}{\text{Total bits per frame}}
\]
\[
\text{Maximum data rate} = \frac{115200 \times 8}{10} = 92160 \text{ bps}
\]

Therefore, the maximum data rate that can be achieved with the given UART configuration is:
\[
92160 \text{ bits per second (bps)}
\]
